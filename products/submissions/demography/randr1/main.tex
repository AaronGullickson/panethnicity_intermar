%%%%%%%%%%%%%%%%%%%%%%%%%%%%%%%%%
% This is a slightly modified template of the one built by
% Steven V. Miller. Information can be found here:
%  http://svmiller.com/blog/2016/02/svm-r-markdown-manuscript/
%
% I added in a few other features.
%
% Here are the options that you can define in the YAML
% header.
%
% fontfamily - self-explanatory
% fontsize - self-explanatory (e.g. 10pt, 11pt)
% anonymous - true/false. If true, names will be supressed and the
%                       text will be double-spaced and ragged
%                       right with a separate page for title and abstract.
% endnotes - true/false. If true, the footnotes will be put in a
%                   section at the end just ahead of the references.
% endfloat - move all tables and figures to the end of the document
% keywords - self-explanatory
% thanks - shows up as a footnote to the title on page 1
% abstract - self explanatory
% appendix - if true, tables and figures will have  in
%                   front
% appendixletter - The letter to append to tables and figures in
%                             appendix
% pagenumber - Put in a number here to get a starting page number
%                         besides 1.
%%%%%%%%%%%%%%%%%%%%%%%%%%%%%%%%%%


\documentclass[11pt,]{article}
\usepackage[left=1in,top=1in,right=1in,bottom=1in]{geometry}
\usepackage{amsmath}
\usepackage{float}
\usepackage{dcolumn}
\usepackage{graphicx}

\newcommand*{\authorfont}{\fontfamily{phv}\selectfont}
\usepackage[]{mathpazo}


\usepackage[T1]{fontenc}
\usepackage[utf8]{inputenc}

\usepackage{abstract}
\renewcommand{\abstractname}{}    % clear the title
\renewcommand{\absnamepos}{empty} % originally center

\providecommand{\tightlist}{%
  \setlength{\itemsep}{0pt}\setlength{\parskip}{0pt}}

\renewenvironment{abstract}
 {{%
    \setlength{\leftmargin}{0mm}
    \setlength{\rightmargin}{\leftmargin}%
  }%
  \relax}
 {\endlist}

\makeatletter
\def\@maketitle{%
  \newpage
%  \null
%  \vskip 2em%
%  \begin{center}%
  \let \footnote \thanks
    {\fontsize{18}{20}\selectfont\raggedright  \setlength{\parindent}{0pt} \@title \par}%
}
%\fi
\makeatother




\setcounter{secnumdepth}{0}

\usepackage{longtable,booktabs}

\title{Patterns of Panethnic Intermarriage in the United States, 1980-2018  }



\author{\Large Aaron Gullickson\vspace{0.05in} \newline\normalsize\emph{University of Oregon, Sociology}  }


\date{}

\usepackage{titlesec}

\titleformat*{\section}{\normalsize\bfseries}
\titleformat*{\subsection}{\normalsize\itshape}
\titleformat*{\subsubsection}{\normalsize\itshape}
\titleformat*{\paragraph}{\normalsize\itshape}
\titleformat*{\subparagraph}{\normalsize\itshape}


\usepackage{natbib}
\setcitestyle{aysep={}}
\bibliographystyle{ajs.bst}


%packages needed by kableExtra
\usepackage{booktabs}
\usepackage{longtable}
\usepackage{array}
\usepackage{multirow}
\usepackage{wrapfig}
\usepackage{float}
\usepackage{colortbl}
\usepackage{pdflscape}
\usepackage{tabu}
\usepackage{threeparttable}
\usepackage{threeparttablex}
\usepackage[normalem]{ulem}
\usepackage{makecell}
\usepackage{xcolor}

%\renewcommand{\refname}{References}
%\makeatletter
%\renewcommand\bibsection{
%    \section*{{\normalsize{\refname}}}%
%}%
%\makeatother

\newtheorem{hypothesis}{Hypothesis}
\usepackage{setspace}

\makeatletter
\@ifpackageloaded{hyperref}{}{%
\ifxetex
  \usepackage[setpagesize=false, % page size defined by xetex
              unicode=false, % unicode breaks when used with xetex
              xetex]{hyperref}
\else
  \usepackage[unicode=true]{hyperref}
\fi
}
\@ifpackageloaded{color}{
    \PassOptionsToPackage{usenames,dvipsnames}{color}
}{%
    \usepackage[usenames,dvipsnames]{color}
}
\makeatother
\hypersetup{breaklinks=true,
            bookmarks=true,
            pdfauthor={},
             pdfkeywords = {panethnicity; intermarriage; assortative mating, ethnic exogamy; immigration},
            pdftitle={Patterns of Panethnic Intermarriage in the United States, 1980-2018},
            colorlinks=true,
            citecolor=blue,
            urlcolor=blue,
            linkcolor=magenta,
            pdfborder={0 0 0}}
\urlstyle{same}  % don't use monospace font for urls

\usepackage{endnotes}


  \usepackage[notablist,nofiglist,noheads,tablesfirst]{endfloat}

\newlength{\normalparindent}
\setlength{\normalparindent}{\parindent}

%prettier captions for figures and tables
%I am making the text of figure captions smaller but not table captions
\usepackage[labelfont=bf,labelsep=period]{caption}
\captionsetup[figure]{font=footnotesize}

\begin{document}

% \pagenumbering{arabic}% resets `page` counter to 1
%


%\pagenumbering{gobble}

% \maketitle

{% \usefont{T1}{pnc}{m}{n}
\setlength{\parindent}{0pt}
\thispagestyle{plain}
{\fontsize{18}{20}\selectfont\raggedright
\maketitle  % title \par

}

{
   \vskip 13.5pt\relax \normalsize\fontsize{11}{12}
\hfill 

}

}







\begin{abstract}

    \hbox{\vrule height .2pt width 39.14pc}

    \vskip 8.5pt % \small

\noindent Intermarriage among ethnic groups belonging to the same panethnic category (e.g.~Asian, Latino) is an important indicator of the strength of panethnicity. Yet, most of the work on panethnic intermarriage uses older samples with significant data limitations. In this article, I use data on recently married couples from the American Community Survey 2014-18 and Census 1980 to analyze the likelihood of ethnic exogamy within the panethnic categories of Latino, East/Southeast Asian, and South Asian. I utilize a counterfactual marriage model that accounts for group size within local marriage markets, eliminates immigrants married abroad from analysis, and controls for birthplace and language endogamy. The results show that birthplace and language diversity are significant barriers to ethnic exogamy among Asians but not Latinos. Once birthplace and language endogamy are held constant, panethnic intermarriage is far more likely among Asians than among Latinos. East/Southeast Asian ethnic exogamy has increased over time, while Latino ethnic exogamy has not. Furthermore, East/Southeast Asian and South Asian intermarriage remains rare, suggesting that panethnic intermarriage among Asians occurs within two separate melting pots.


\vskip 8.5pt \noindent \emph{Keywords}: panethnicity; intermarriage; assortative mating, ethnic exogamy; immigration \par

    \hbox{\vrule height .2pt width 39.14pc}



\end{abstract}


\vskip 6.5pt

\noindent \newpage\doublespacing\raggedright\setlength{\parindent}{\normalparindent} \hypertarget{introduction}{%
\section{Introduction}\label{introduction}}

Social scientists treat intermarriage as a prime indicator of the underlying social distance and strength of boundaries separating groups. Because marriage is one of the most intimate kinds of social relationships, widespread willingness to marry across group lines is seen as a key signifier that the salience of the boundaries separating those groups is weak \citep{gordon_assimilation_1964}.

An increase in intermarriage across one boundary is often treated as an indication that this boundary is dissolving or blurring, but it can also indicate reshaping of existing boundaries. Widespread intermarriage among European ethnic groups in the mid-twentieth century contributed to the breakdown of salient ethnic divisions between these populations but also helped to re-consolidate a sense of collective whiteness \citep{lieberson_many_1988, alba_ethnic_1990, jacobsen_whiteness_1998}. Similarly, spurred by resurgent immigration to the US since the 1960s, intermarriage is seen as a key benchmark to gauge panethnic affinity among Asian and Latino ethnic groups today.

Panethnicity is defined by \citet{okamoto_panethnicity_2014a} as ``the construction of a new categorical boundary through the consolidation of ethnic, tribal, religious, or national groups.'' Within the literature on ethnoracial boundary formation, panethnicity is treated as a form of boundary expansion in which the salient boundary shifts from one level to another within a nested hierarchy of possible identification \citep{wimmer_making_2008}.\footnote{The distinction between a racial and an ethnic group is often fuzzy and debated among scholars \citep{brubaker_ethnicity_2009, valdez_racial_2017, lewis_race_2017}. For analytical clarity, I use the term ``ethnoracial'' to refer to any group that may be identified either along racial or ethnic lines in popular practice, ``racial group'' to refer to the five major groups of White, Black, Indigenous, Asian, and Latino that constitute the highest level in the nested hierarchy of ethnoracial differences, and ``ethnic group'' to refer to different sub-populations among Asians and Latinos that are primarily defined in terms of national origin, such as Chinese, Korean, Mexican, and Colombian. Ethnic differentiation within the same national origin group exists as well, but is largely unmeasurable in the data that I use here.} From this boundary perspective, ethnoracial categories are not fixed, stable, or primordial, but can shift over time in response to historical events, as they have for the ethnoracial groups in the US that we now collectively view as White and Black. Understanding how such processes are playing out for contemporary Asian and Latino ethnic groups informs us about how social boundaries might shift in the future. Such shifts are consequential for issues of racial inequality, assimilation, political change, and even how we measure race and ethnicity.

The earliest work on panethnicity focused on collaboration among ethnically based organizations and movements \citep{lopez_panethnicity_1990, okamoto_theory_2003}. However, these authors also noted the importance of examining patterns of marriage to better understand ``interpersonal'' panethnicity \citep[pp.~167-168]{espiritu_asian_1993}. Early attempts to address this research question looked at outmarriage percentages which generally show the effects of group size more than underlying affinity \citep{shinagawa_asian_1996}. However, shortly after the turn of the century, several more sophisticated studies were published that generally found stronger evidence of panethnicity among Asians than Latinos \citep{qian_asian_2001, rosenfeld_salience_2001, qian_latinos_2004, fu_how_2007a, qian_crossing_2012}.

This prior work on intermarriage has contributed greatly to our understanding of the phenomenon of panethnicity, but is also considerably out of date. Prior work primarily used data from Census 1980, 1990, or 2000. The two later data sources have no information on marriage timing which limits researchers' ability to exclude immigrants married abroad from analysis. Furthermore, much of this work relies upon an examination of panethnicity among the relatively few Asian and Latino groups that were numerically large enough in historical data to sustain an analysis. Most prior work has either excluded South Asians entirely or only included a single South Asian ethnonational group. Diversification in country of origin among Asian and Latino immigrants makes possible a broader analysis using contemporary data.

Furthermore, prior research has paid insufficient attention to the role of birthplace and language endogamy in shaping patterns of panethnic intermarriage. Generally, individuals prefer partners from the same birthplace and who speak the same language. The considerable diversity in birthplace and language both within and between Asian and Latino ethnic groups has complex effects on the tendency toward panethnic intermarriage. Prior work has largely tried to address this issue indirectly by including an immigrant-native comparison, but this comparison does not adequately capture the complexity of birthplace and language diversity.

In this article, I broaden our understanding of panethnic intermarriage by applying a new modeling approach to recent data from the 2014-2018 American Community Survey. This data includes a wide variety of Asian and Latino ethnic groups, including several South Asian ethnic groups. Additionally, I conduct a parallel analysis of Census 1980 data to provide a measure of the change in panethnic intermarriage over time. Both of these data sources include timing of marriage which resolves several methodological difficulties with prior work. I use a relatively novel counterfactual marriage model technique \citep{gullickson_counterfactual_2021} that allows me to easily adjust for differences in the size and spatial distribution of ethnoracial groups. Furthermore, this model allows me to account for the role of birthplace and language endogamy.

Using this model, I estimate the strength of panethnicity for Latinos, East/Southeast Asians, and South Asians across the two time periods and compare this to patterns of White and Black outmarriage for these groups. I find that patterns of panethnicity are very different for Asian and Latino ethnic groups. For Asians, panethnic affinity is very strong among East/Southeast Asian ethnic groups and among South Asian ethnic groups, but there is little evidence of panethnicity between these two populations. Furthermore, Asian panethnic intermarriage has also become more common over time. For Latinos, panethnic affinity is lower overall and in some cases is less likely than outmarriage to non-Latino racial groups.

\hypertarget{panethnicity-in-marriage}{%
\section{Panethnicity in Marriage}\label{panethnicity-in-marriage}}

Panethnic affinity may be driven by both cultural and structural factors \citep{lopez_panethnicity_1990}. Cultural factors such as shared language and religion create affinity across ethnic or national boundaries. Structural factors such as economic and occupational similarity, spatial proximity, and the degree of racialization (the tendency of outsiders to treat all members of the panethnic category as a monolithic racial group) may also play a role in encouraging or discouraging panethnic affinity. The distinction between cultural factors and racialization has been particularly important in differentiating the experiences of Asians and Latinos. Latinos share cultural features such as language and religion that should promote panethnicity while Asians generally do not. On the other hand, racialization of Asians as a singular group tends to be stronger than for Latinos, who are more phenotypically diverse and racially ambiguous \citep{lopez_panethnicity_1990, kibria_construction_1997, qian_latinos_2004}.

Several important studies of panethnic intermarriage were performed shortly after the turn of the century, using data from the 1980 and 1990 US Census \citep{qian_asian_2001, rosenfeld_salience_2001, qian_latinos_2004, fu_how_2007a}. Both \citet{rosenfeld_salience_2001} and \citet{fu_how_2007a} explicitly compare the tendency toward panethnicity among Asians and Latinos and find a stronger tendency toward panethnicity among Asians. \citet{qian_asian_2001} and \citet{qian_latinos_2004} use similar methods to examine panethnicity among Asians and Latinos, respectively, and find substantial evidence of panethnic affinity, particularly among the native-born. Using slightly more recent data from Census 2000, \citet{qian_crossing_2012} estimate the likelihood of panethnic intermarriage for Mexicans, Puerto Ricans, Chinese, and Filipinos. Their results show similar levels of panethnicity across all four groups, and that panethnicity is most likely among the native-born and those immigrants who came to the US at an early age.

Some of these studies also compare the likelihood of interethnic marriage to the likelihood of interracial marriage with Whites and Blacks \citep{qian_asian_2001, fu_how_2007a, qian_crossing_2012}. In general, scholars have found that Asians are more likely to marry interethnically than to marry Whites or Blacks. Latinos are also less likely to marry Blacks than to marry interethnically, but in some cases, depending on generation and the ethnic group under consideration, are more likely to marry Whites than to marry interethnically.

Although this prior work has enriched our understanding of panethnic intermarriage, it also suffers from well-acknowledged methodological difficulties. All of the studies cited above use log-linear models to adjust for differences in group size. This approach allows researchers to properly estimate the underlying affinity between groups, apart from differential exposure to potential partners due to group size. However, basic log-linear models will not address differences in spatial settlement patterns among groups. Most individuals will find a partner within a local, rather than a national, context, and ethnoracial groups are not evenly distributed across the US. Without adjusting for these differences in geography, scholars will tend to underestimate the degree of ethnoracial exogamy \citep{harris_how_2005}. This issue is particularly problematic when studying panethnicity, because Latino and Asian ethnic groups have historically settled in different parts of the US \citep{massey_geographic_2008}. Some of the prior work on panethnicity attempts to address this issue. \citet{fu_how_2007a} includes Census division as a parameter in log-linear models, while \citet{rosenfeld_salience_2001} estimated log-linear models separately within a few specific metropolitan areas. However, most of the prior work makes no adjustment for this important issue.

Ideally, researchers seek to capture the incidence of intermarriage by examining recently formed marriages. However, most large-scale data has lacked information on marriage timing and so researchers typically resort to restricting the age of spouses to create semi-incidence measures with unknown biases. The problems arising from lack of marriage timing, are compounded by the issue of immigrants married abroad (IMA). Without information on both timing of marriage and timing of immigration to the US, it is impossible to completely remove IMA from the analysis. Because IMA were mostly married in their country of origin, they will bias estimates toward ethnic endogamy \citep{hwang_problem_1990}. Researchers have used a variety of sample restrictions to minimize this problem, but without information on marriage and migration timing, it cannot be fully eliminated.

A final issue in measuring panethnic intermarriage is the consideration of what ethnic groups to include. Historically, the most populous ethnic groups were Mexicans, Puerto Ricans, and Cubans among Latinos and Chinese, Filipinos, Japanese, Koreans, and Vietnamese among Asians. Prior work has focused on some subset of these groups because they are the numerically largest and can be clearly identified in earlier Census data. However, limiting analysis to only a few ethnic groups may distort our understanding of panethnic intermarriage across all groups.

Related to this issue, most prior work has not addressed the issue of South Asian panethnic intermarriage. The placement of South Asians within the panethnic category of Asian is contested at best. Individuals from the Indian subcontinent are treated as part of the panethnic Asian group by governmental agencies such as the Census Bureau, but research has shown that in everyday practice they are generally treated as ``ambiguous non-whites'' \citep{kibria_not_1996, morning_racial_2001, schachter_finding_2014}. Consistent with this view, prior work on intermarriage among Asian Indians has shown that this group has relatively low rates of intermarriage with other (non-South) Asian populations \citep{qian_asian_2001, lichter_whom_2015a}. However, due to sample size limitations, Asian Indians are the only South Asian population included in prior analyses, making it impossible to explore the extent of panethnicity among South Asian ethnic groups.

\hypertarget{the-role-of-birthplace-and-language-endogamy}{%
\subsection{The Role of Birthplace and Language Endogamy}\label{the-role-of-birthplace-and-language-endogamy}}

Birthplace and language endogamy affect the likelihood of panethnic intermarriage in complex ways. Individuals often prefer partners from the same birthplace due to the shared cultural understandings that arise from being born and raised in a particular place. Similarly, people are more likely to marry individuals who speak the same primary language. We might expect birthplace endogamy to lower the likelihood of panethnicity for both Asians and Latinos, because all Asian and Latino ethnic groups have a substantial foreign-born sub-population and these members would prefer to marry someone from the same place of birth. On the other hand, language endogamy might affect Asian and Latino panethnic intermarriage differently, because Latino ethnic groups share a common language while Asian ethnic groups do not.

However, the actual effect of birthplace and language endogamy on panethnic intermarriage is considerably more complex, because ethnic groups themselves are diverse in terms of birthplace and language. For example, among adults recently married or single in the 2014-18 American Community Survey data detailed below, 66\% of Japanese respondents and 41\% of Filipino respondents spoke English as a primary language. For those Japanese and Filipino individuals who both speak English, language endogamy will actually encourage panethnic intermarriage relative to ethnic endogamy with members of their own group who speak Japanese and Tagalog, respectively. Similarly, 31\% of Mexicans in the same sample spoke English as their primary language, but only 13\% of Dominicans. In this case, however, Mexicans and Dominicans who do not speak English typically both speak Spanish which will encourage panethnic intermarriage when such potential partners are paired.

In general, in order for language and birthplace endogamy to serve as a barrier to panethnic intermarriage, the diversity in language and birthplace must be greater within the panethnic category than within the corresponding ethnic group. The strength of the barrier will depend on how much more diverse the panethnic category is than the ethnic group. To illustrate this issue, I calculate a measure of language/birthplace diversity for unmarried and recently married members of each Asian and Latino ethnic group in the Census 1980 and American Community Survey detailed below. Specifically, I use the Blau index \citep{blau_inequality_1977} to measure language/birthplace diversity within each Asian and Latino ethnic group and to measure this same diversity within each panethnic category. The Blau index of diversity (\(D\)) is calculated as: \[D=1-\sum_{i=1}^R p_i^2\] where \(p_i\) is the proportion of the total population belonging to group \(i\) and \(R\) is the total number of groups. The Blau index has an intuitive interpretation as the probability that two randomly drawn members from the population belong to the same group.

\begin{figure}
\centering
\includegraphics{main_files/figure-latex/diversity-pan-bar-1.pdf}
\caption{\label{fig:diversity-pan-bar}Birthplace and language diversity within Asian and Latino ethnic groups and panethnic categories. Diversity is measured by the Blau index which gives the probability that two randomly selected members of the group do not share the same birthplace/language. Results are based upon alternate partners from each data source. Diversity among South Asian is only measurable in the later data source due to data limitations.}
\end{figure}

Figure \ref{fig:diversity-pan-bar} shows the Blau index for language and birthplace across East/Southeast Asian, South Asian, and Latino ethnic and panethnic categories in the two time periods. To simplify presentation, I compare the diversity across the panethnic category to the mean diversity within each specific ethnic group (weighted by group size). The likelihood of panethnic intermarriage will be reduced in cases where panethnic diversity is greater than average diversity within ethnic groups.

Significant language and birthplace diversity is observable within specific ethnic groups. For example, the average diversity in birthplace within specific Latino ethnic groups in the full ACS data is about 50\%, indicating that two randomly determined members of the same Latino ethnic group would not share the same birthplace about 50\% of the time.

Although within ethnic group diversity is substantial, panethnic diversity in language and birthplace is greater than within ethnic group diversity for all three panethnic categories in both time periods. However, these differences are small for Latinos, while they are quite large for Asians. What differences exist among Latino ethnic groups have also diminished over time, whereas they have grown for East/Southeast Asians.

Overall, Figure \ref{fig:diversity-pan-bar} shows that language and birthplace endogamy are important barriers to panethnic intermarriage among Asians, but not among Latinos. Ideally, to estimate the underlying affinity for panethnic intermarriage, we want a model that can separate out barriers to intermarriage resulting from language and birthplace endogamy. Such a model indicates how much panethnic intermarriage we would expect in a situation in which all Asians and Latinos are fully acculturated to the US (i.e.~born in the US and speak English as their primary language).

In the analysis that follows, I address many of these prior weaknesses using a counterfactual marriage model to estimate the likelihood of panethnic intermarriage. The data from Census 1980 and the American Community Survey 2014-18 provide information on marriage timing which allows me to limit analysis to recent marriages while removing immigrants married abroad. The model framework allows me to account for differences in group size in local marriage markets while easily incorporating controls for birthplace and language endogamy.

\hypertarget{data-and-methods}{%
\section{Data and Methods}\label{data-and-methods}}

The data for this analysis are derived from the microdata sample of the 1980 US Census and the American Community Survey (ACS) for 2014-2018. The ACS is an annual 1-in-100 survey of the United States population, conducted by the United States Census Bureau. The 1980 US Census is the last Census dataset to include information about marriage timing, before it was re-included on the 2008 ACS. To increase sample size for smaller groups, I pool five years of ACS data from 2014-2018. Both data sources were extracted from the Integrated Public Use Microdata Series (IPUMS) system \citep{ruggles_ipums_2020}.

In both data sources, I restrict the analysis to all opposite-sex marriages formed in the previous five years that were a first marriage for both partners. This restriction is necessary for comparison because the Census 1980 only recorded marriage timing relative to first marriage and does not include same-sex unions. To avoid capturing marriages occurring in different local contexts, I also remove marriages in which at least one of the partners migrated to the US or across state lines in the last five years.

To measure the likelihood of panethnic intermarriage, I use a modeling technique that compares actual marriages to alternate marriages that were not formed \citep{gullickson_counterfactual_2021}. For each marriage, I construct a choice set of one real union and twenty fictional unions. Fictional unions are created by sampling alternate partners for one randomly determined spouse from a pool of potential partners. I then use a conditional logit model to predict how partner characteristics relate to the likelihood of observing the true union, as follows: \[P_{ij}=\frac{e^{\mathbf{x}_{ij}\beta}}{\sum_{k=1}^J e^{\mathbf{x}_{ik}\beta}}\] where \(P_{ij}\) is the probability that union \(j\) within choice set \(i\) is the actual union. \(J\) is the total number of unions in the choice set. The vector \(\mathbf{x}_{ij}\) defines the characteristics of the union and the \(\beta\) vector provides estimated log-odds ratios indicating how the odds of an actual union change with \(\mathbf{x}_{ij}\). The model is estimated as a fixed-effects logistic regression model with fixed effects for each choice set.

This approach has been used previously to examine patterns of intermarriage \citep{dalmia_empirical_2001, jepsen_empirical_2002, nielsen_educational_2009, qian_marriage_2018} and friendship choices \citep{zeng_preference_2008}. This model specification has several advantages over more traditional log-linear models. Like a log-linear model, this model intrinsically accounts for differences in group size through the sampling procedure, but it also takes into account the unmarried population which is ignored in a log-linear model. Furthermore, the linear structure of the conditional logit model can accommodate a variety of quantitative and categorical control variables much easier than a log-linear model.

Arguably, the most important advantage of this modeling approach is that the researcher can specify additional restrictions in the sampling of potential partners. I utilize that feature to restrict potential partners to a locally defined marriage market, which addresses the issue of spatial dissimilarity among ethnoracial groups. Ideally, I would use metropolitan area to identify marriage markets, but not all metropolitan areas are identifiable in the public use Census 1980 and ACS data, due to confidentiality concerns. Furthermore, some respondents do not live in metropolitan areas. Therefore, I use metropolitan area as the marriage market identifier for individuals where it can be identified and otherwise I use the state of residence.\footnote{Metropolitan area is determined by IPUMS based on the county group geography in Census 1980 and the Public Use Microdata Area (PUMA) geography in the ACS. IPUMS only identifies metropolitan areas in cases where this geographic identifier unambiguously indicates residence within the metropolitan area.} I can identify 255 metropolitan areas in Census 1980 and 260 metropolitan areas in the ACS data. Although I am not able to identify all metropolitan areas, most major metropolitan areas are identified in both data sources.\footnote{Because the identified metropolitan areas are not identical across time periods, some bias may be present when analyzing change over time. To address this issue, I conducted a sensitivity test using state as the marriage market for all respondents. The results for this sensitivity analysis are substantively very similar to those shown here. The results of this sensitivity analysis are provided in the supplementary materials.} From within a given marriage market, I draw alternate partners from among all unmarried adults as well as individuals who were married in the previous five years, with the restriction that all alternate partners must not have migrated to the US or across state lines in the last five years.

Because this approach relies upon a random sample of alternate partners, results will vary each time this sampling procedure is performed. To account for this added uncertainty, I conduct the analysis using three different analytical samples. I then pool \(\beta\) estimates and standard errors for parallel models using the same methods as those employed for multiple imputation \citep{rubin_multiple_1987, gullickson_counterfactual_2021}.

\hypertarget{measuring-ethnoracial-exogamy}{%
\subsection{Measuring Ethnoracial Exogamy}\label{measuring-ethnoracial-exogamy}}

I measure ethnoracial categories by a combination of the race and Hispanicity questions in both data sources. Table \ref{tab:sample-size-table} shows the sample size for all ethnoracial groups included in the analysis. I classify non-Asian and non-Latino respondents as White, Black, or American Indian/Alaska Native (AIAN). Due to small sample size, I exclude Pacific Islanders, multiracial respondents, and non-Latino others. The race question includes several Asian nationalities (e.g.~Chinese, Korean, Filipino) as well as a write-in response for ethnic/national identities not captured by the existing categories. Similarly, the Hispanicity question includes major Latino ethnic groups (e.g.~Mexican, Puerto Rican, Cuban) as well as a write-in option. As indicated in Table \ref{tab:sample-size-table}, the Census 1980 data includes a far more limited set of identifiable ethnic groups for both Asians and Latinos than the ACS data. To address this restriction, I estimate two different kinds of models for the ACS data. First, I restrict the data only to those ethnic groups that were available in the Census 1980 data to allow for direct comparison over time. I then estimate models using the full ACS data, but these models are not directly comparable to the Census 1980 models.

Table \ref{tab:sample-size-table} also shows how I define panethnic groups of East/Southeast Asian, South Asian, and Latino. I separate Asians into two separate panethnic blocs because prior work suggests social distance between these groups \citep{kibria_not_1996, morning_racial_2001, schachter_finding_2014}. I estimate the likelihood of intermarriage between these two blocs in all models to test whether this separation is justified empirically. Because Asian Indians are the only identifiable South Asian ethnic group in Census 1980, I cannot estimate panethnic parameters over time for South Asians. Thus, when analyzing change over time, I focus exclusively on East/Southeast Asians and Latinos.

\begin{table}

\caption{\label{tab:sample-size-table}Sample size of marriages and alternate partners by data source and ethnoracial category.}
\centering
\begin{tabu} to \linewidth {>{\raggedright\arraybackslash}p{6.5cm}>{\raggedleft}X>{\raggedleft}X}
\toprule
Category & Census 1980 & ACS 2014-2018\\
\midrule
\addlinespace[0.3em]
\multicolumn{3}{l}{\textbf{Marriages}}\\
\hspace{1em}Marriages in the previous five years & 285,523 & 503,348\\
\addlinespace[0.3em]
\multicolumn{3}{l}{\textbf{Alternate Partners}}\\
\hspace{1em}White & 2,926,629 & 4,368,640\\
\hspace{1em}Black & 535,993 & 887,837\\
\hspace{1em}American Indian/Alaska Native & 24,304 & 73,760\\
\hspace{1em}Latino & 162,526 & 859,250\\
\hspace{1em}\hspace{1em}Mexican & 113,648 & 560,124\\
\hspace{1em}\hspace{1em}Puerto Rican & 35,444 & 97,695\\
\hspace{1em}\hspace{1em}Cuban & 13,434 & 39,473\\
\hspace{1em}\hspace{1em}Salvadorian &  & 32,947\\
\hspace{1em}\hspace{1em}Dominican &  & 29,201\\
\hspace{1em}\hspace{1em}Guatemalan &  & 19,449\\
\hspace{1em}\hspace{1em}Colombian &  & 18,826\\
\hspace{1em}\hspace{1em}Honduran &  & 11,926\\
\hspace{1em}\hspace{1em}Peruvian &  & 10,548\\
\hspace{1em}\hspace{1em}Ecuadorian &  & 10,226\\
\hspace{1em}\hspace{1em}Nicaraguan &  & 7,605\\
\hspace{1em}\hspace{1em}Argentinian &  & 4,619\\
\hspace{1em}\hspace{1em}Venezuelan &  & 4,404\\
\hspace{1em}\hspace{1em}Panamanian &  & 3,960\\
\hspace{1em}\hspace{1em}Chilean &  & 2,531\\
\hspace{1em}\hspace{1em}Costa Rican &  & 2,399\\
\hspace{1em}\hspace{1em}Bolivian &  & 1,814\\
\hspace{1em}\hspace{1em}Uruguayan &  & 1,046\\
\hspace{1em}\hspace{1em}Paraguayan &  & 457\\
\hspace{1em}East and Southeast Asian & 30,432 & 199,180\\
\hspace{1em}\hspace{1em}Chinese & 9,980 & 63,621\\
\hspace{1em}\hspace{1em}Filipino & 6,557 & 48,254\\
\hspace{1em}\hspace{1em}Vietnamese & 331 & 29,119\\
\hspace{1em}\hspace{1em}Korean & 2,066 & 24,163\\
\hspace{1em}\hspace{1em}Japanese & 11,498 & 15,023\\
\hspace{1em}\hspace{1em}Cambodian &  & 4,880\\
\hspace{1em}\hspace{1em}Hmong &  & 4,563\\
\hspace{1em}\hspace{1em}Laotian &  & 3,762\\
\hspace{1em}\hspace{1em}Thai &  & 3,429\\
\hspace{1em}\hspace{1em}Burmese &  & 1,156\\
\hspace{1em}\hspace{1em}Indonesian &  & 980\\
\hspace{1em}\hspace{1em}Malaysian &  & 230\\
\hspace{1em}South Asian & 2,882 & 41,222\\
\hspace{1em}\hspace{1em}Asian Indian & 2,882 & 34,446\\
\hspace{1em}\hspace{1em}Pakistani &  & 4,785\\
\hspace{1em}\hspace{1em}Bangladeshi &  & 1,402\\
\hspace{1em}\hspace{1em}Sri Lankan &  & 589\\
\bottomrule
\end{tabu}
\end{table}

In all models, patterns of ethnoracial exogamy, including panethnic intermarriage, are measured by a set of gender-symmetric dummy variables where the reference category is an ethnoracially endogamous union (e.g.~a White-White or Chinese-Chinese marriage). Although substantial gender asymmetry exists in intermarriage for several important combinations \citep{xie_demographic_2000, gullickson_black_2006}, the use of gender-symmetric terms more closely matches my goal of estimating social boundaries between groups. Such social boundaries are best measured by averaging across gender combinations, which is accomplished by gender symmetric terms. Furthermore, given the large number of parameters involved in some models, gender asymmetric terms would be impossible to fit in many cases.

Even using gender-symmetric terms, the least parsimonious model would simply include every possible combination of categories resulting in 703 separate dummy variables. The sheer number of variables required and resulting data sparseness makes such a model unfeasible. Instead, I use two different approaches to more parsimoniously capture the likelihood of panethnic intermarriage and ethnoracial exogamy. The coding scheme for the first approach is illustrated in Table \ref{tab:block-diagram}. Two simplifications are made within this scheme. First, I model all ethnically exogamous unions within the same panethnic category using a single dummy variable that identifies the union as a panethnic intermarriage. For example, a Japanese-Korean union and a Chinese-Korean union would both be classified as East/Southeast Asian ethnic exogamy. This approach allows me to estimate the average likelihood of ethnic exogamy within a panethnic category, at the cost of neglecting potential heterogeneity in this likelihood between certain ethnic combinations. Second, when analyzing ethnoracial exogamy outside of panethnic groups, I use the larger panethnic categories of Latino, East/Southeast Asian, and South Asian. For example, a union between a White and Mexican person and between a White and a Guatemalan person would both be classified as Latino/White exogamy. These two simplifications reduce the number of required parameters to 17.

\begin{table}

\caption{\label{tab:block-diagram}A schematic representation of the coding of ethnoracial exogamy for simplified models, using three example ethnicities for Asian and Latino populations. The table shows a crosstabulation of each partner's race. All parameters are gender-symmetric, so I only show the parameters below the diagonal. The reference category is an ethnoracially endogamous union. Each cell indicates the particular dummy variable that is applied to a given case. The terms measuring panethnicity are shown in bold.}
\centering
\resizebox{\linewidth}{!}{
\begin{tabular}[t]{l>{}c>{}c>{}c>{}c>{}c>{}c>{}c>{}c}
\toprule
\multicolumn{3}{c}{ } & \multicolumn{3}{c}{Asian} & \multicolumn{3}{c}{Latino} \\
\cmidrule(l{3pt}r{3pt}){4-6} \cmidrule(l{3pt}r{3pt}){7-9}
  & White & Black & Chinese & Japanese & Korean & Mexican & Cuban & Puerto Rican\\
\midrule
White & \em{(ref.)} &  &  &  &  &  &  & \\
Black & B/W & \em{(ref.)} &  &  &  &  &  & \\
\addlinespace[0.3em]
\multicolumn{9}{l}{\textbf{Asian}}\\
\hspace{1em}Chinese & A/W & A/B & \em{(ref.)} &  &  &  &  & \\
\hspace{1em}Japanese & A/W & A/B & \textbf{PE-A} & \em{(ref.)} &  &  &  & \\
\hspace{1em}Korean & A/W & A/B & \textbf{PE-A} & \textbf{PE-A} & \em{(ref.)} &  &  & \\
\addlinespace[0.3em]
\multicolumn{9}{l}{\textbf{Latino}}\\
\hspace{1em}Mexican & L/W & L/B & L/A & L/A & L/A & \em{(ref.)} &  & \\
\hspace{1em}Cuban & L/W & L/B & L/A & L/A & L/A & \textbf{PE-L} & \em{(ref.)} & \\
\hspace{1em}Puerto Rican & L/W & L/B & L/A & L/A & L/A & \textbf{PE-L} & \textbf{PE-L} & \em{(ref.)}\\
\bottomrule
\multicolumn{9}{l}{\rule{0pt}{1em}\textit{Notes: }}\\
\multicolumn{9}{l}{\rule{0pt}{1em}B/W=Black/White, A/W=Asian/White, A/B=Asian/Black, L/W=Latino/White, L/B=Latino/Black,}\\
\multicolumn{9}{l}{\rule{0pt}{1em}L/A=Latino/Asian, PE-A=Panethnic Asian, PE-L=Panethnic Latino}\\
\end{tabular}}
\end{table}

To address the potential for this method to miss important heterogeneity in the likelihood of ethnoracial exogamy, I also fit a less parsimonious model to the ACS data. In this model, I code each specific combination of ethnic groups within the same panethnic category using a separate dummy variable. Additionally, to determine whether the tendency to outmarry with Whites or Blacks varies among ethnic groups within the same panethnic category, I treat each combination of an ethnic group with the White and Black categories as a separate variable. I restrict the ethnic groups to those for which I can fit a model without problems of sparseness and model non-convergence, determined by sequentially including ethnic groups by size. The final model includes the five East/Southeast Asian ethnic groups available in the Census 1980 data (Chinese, Filipino, Vietnamese, Korean, and Japanese) and the ten largest Latino groups, with the exception of Hondurans. I could not fit these models for South Asians because of the small size of non-Asian Indian groups in the South Asian category. This model includes 85 separate exogamy terms that better capture heterogeneity within panethnic categories but at a significant cost to parsimony.

Regardless of the specific model, the coefficient for each ethnoracial exogamy term has a similar interpretation. The exponentiated coefficient gives the ratio of the odds of a union between the two specified ethnoracial groups relative to the odds of ethnoracial endogamy. Values below one indicate that exogamy is less likely than endogamy. A relatively lower odds ratio indicates lower likelihood of this form of exogamy relative to other forms of exogamy. These odds ratios represent the likelihood of intermarriage net of group size differences in partner availability. These group size differences are accounted for by the sampling procedure which will draw alternate partners from different groups in proportion to their size in the designated marriage market.

Each odds ratio is also unaffected by the degree of ethnoracial exogamy to other groups. For a given focal group, a high odds ratio of exogamy to one group does not entail that the odds ratio of exogamy to other groups must necessarily be low. Theoretically, for example, the odds ratio of exogamy to all outgroups could equal one, indicating no preference for endogamy and that partnering was conducted randomly with regard to ethnorace.

\hypertarget{additional-variables}{%
\subsection{Additional Variables}\label{additional-variables}}

The model used here allows for the easy incorporation of a variety of quantitative and categorical variables. I utilize that feature to account for birthplace and language endogamy when estimating the likelihood of panethnic intermarriage, as well as age and educational differences between partners.

I account for language and birthplace endogamy with simple dummy variables indicating whether the two potential partners share the same primary language or birthplace, respectively. Primary language is determined by what language the respondent reported speaking at home. Because this variable is measured after a marriage occurred, it may somewhat overestimate language endogamy among respondents.

Birthplace endogamy is complicated by the ``1.5'' generation -- individuals born outside of the United States but who migrated as children and whose formative experiences are partially defined by acculturation within the US. I consider three possibilities for coding the birthplace endogamy of such individuals, organized by the assumed degree of acculturation. First, such individuals could be considered birthplace endogamous only with a person from the same actual birthplace. Second, such individuals could be considered birthplace endogamous with either a person from their actual birthplace or a person born in the USA. Third, such individuals could be considered only birthplace endogamous with a person born in the USA.

To test the accuracy of these three scenarios, I fit models using each coding scheme to the Census 1980 and ACS data. Following \citet{rumbaut_ages_2004a}, I also divide the 1.5 generation into a ``1.75'' generation (those who arrived in the US before the age of 6), a ``1.5'' generation (those who arrived in the US between the ages of 6-12), and a ``1.25'' generation (those who arrived in the US between the ages of 13-17). For these three groups, I consider every possible combination of coding such that earlier generations are not more acculturated than later generations (e.g.~the 1.25 generation could not be ``both'' in cases where the 1.5 generation is ``birthplace only''). Table \ref{tab:deviance-bendog} shows the model fit by deviance of all ten possible combinations for both data sources. Lower deviance indicates better fit.

\begin{table}

\caption{\label{tab:deviance-bendog}Model fit to Census 1980 and ACS data using different specifications of birthplace endogamy for 1.25, 1.5, and 1.75 generations. Minimum deviance is shown in bold.}
\centering
\begin{tabu} to \linewidth {>{\raggedright}X>{\raggedright}X>{\raggedright}X>{\raggedleft}X>{\raggedleft}X}
\toprule
\multicolumn{3}{c}{Generation} & \multicolumn{2}{c}{Model Deviance} \\
\cmidrule(l{3pt}r{3pt}){1-3} \cmidrule(l{3pt}r{3pt}){4-5}
1.75 & 1.5 & 1.25 & Census 1980 & ACS 2014-18\\
\midrule
Birthplace & Birthplace & Birthplace & \textbf{1,122,736} & 1,850,594\\
Both & Birthplace & Birthplace & 1,122,758 & 1,849,753\\
USA & Birthplace & Birthplace & 1,122,856 & 1,850,491\\
Both & Both & Birthplace & 1,122,775 & \textbf{1,849,202}\\
USA & Both & Birthplace & 1,122,851 & 1,849,714\\
USA & USA & Birthplace & 1,123,002 & 1,851,089\\
Both & Both & Both & 1,122,932 & 1,850,676\\
USA & Both & Both & 1,122,968 & 1,850,894\\
USA & USA & Both & 1,123,013 & 1,851,576\\
USA & USA & USA & 1,122,968 & 1,852,855\\
\bottomrule
\multicolumn{5}{l}{\rule{0pt}{1em}\textit{Notes: }}\\
\multicolumn{5}{l}{\rule{0pt}{1em}Age of US arrival by generation is 0-5 (1.75), 6-12 (1.5), and 13-17 (1.25). All models include}\\
\multicolumn{5}{l}{\rule{0pt}{1em}controls for educational differences, age differences, ethnoracial endogamy, and language}\\
\multicolumn{5}{l}{\rule{0pt}{1em}endogamy.}\\
\end{tabu}
\end{table}

For both time periods, the most preferred models use birthplace only coding for those respondents who entered the US after age 12. The USA only option was not preferred for any group in either time period. In the Census 1980 data, the most preferred model treated all three groups as first generation individuals. The more recent ACS data preferred a model in which both the 1.75 and 1.5 generations were considered birthplace endogamous with partners either from the actual birthplace or the USA. The shift over time suggests greater acculturation of the 1.5 and 1.75 generation today than in 1980. For all subsequent models, I code birthplace endogamy according to the best-fitting model for each data source from Table \ref{tab:deviance-bendog}.

All models include controls for age and educational differences between potential partners. Age differences are modeled by taking the numerical age difference between partners and its square. Educational differences are modeled using educational crossing parameters \citep{schwartz_trends_2005} with four categories of education (less than a high school diploma, a high school diploma, some college education, and at least a four-year college degree). I also include parameters that measure the likelihood of female educational hypergamy (marrying higher in education) and female educational hypogamy (marrying lower in education).

\hypertarget{results}{%
\section{Results}\label{results}}

I begin the results section by illustrating the results from the models which use a single ethnic exogamy term for each panethnic group. I show how the strength of these ethnic exogamy terms changes with the inclusion of controls for birthplace and language endogamy, followed by an examination of how the strength of panethnicity has changed over time and in relation to other forms of ethnoracial exogamy.\footnote{The analysis of change over time relies upon two time points where the data are sufficient to identify recent marriages. It is possible that patterns of ethnoracial exogamy changed in non-linear ways in the interim between these periods.} I then move to an analysis of models that examine heterogeneity in panethnicity among ethnic groups belonging to the same panethnic category for the later time period. These models can reveal nuances ignored by the more parsimonious approach.

I use graphical visualizations to present important results from all models. Results are presented as the ratio of the odds of a particular kind of exogamy to the odds of ethnoracial endogamy. An odds ratio of one indicates no social distance for a given kind of exogamy. Full results upon which figures are based are available in the supplementary materials.

\hypertarget{the-strength-of-panethnicity}{%
\subsection{The Strength of Panethnicity}\label{the-strength-of-panethnicity}}

Figure \ref{fig:exog-lolly} shows the estimated odds of ethnic exogamy relative to ethnic endogamy for all three panethnic categories in both time periods. As these odds ratios approach one, the odds of ethnic exogamy (e.g.~a Chinese person marrying a Japanese person) increase relative to ethnic endogamy (e.g.~a Japanese person marrying a Japanese person).

\begin{figure}
\centering
\includegraphics{main_files/figure-latex/exog-lolly-1.pdf}
\caption{\label{fig:exog-lolly}The odds of ethnic exogamy for Latinos, East/Southeast Asians, and South Asians over time and model specification. The baseline model controls for age and educational differences.}
\end{figure}

The baseline model only adjusts for age and educational differences between potential partners. I then add birthplace and language endogamy separately and together to determine the effects of these variables on the odds of ethnic exogamy.

In the baseline model, the odds of ethnic exogamy are similar for all three groups. Ethnic exogamy is about 25\% as likely as ethnic endogamy. The results also show a very slight decline in the odds of ethnic exogamy from 1980 to 2014-2018 when restricted to ethnic groups that were available in the 1980 data.

For Latinos, controlling for birthplace and language endogamy increases the overall odds of ethnic exogamy only very slightly, with birthplace endogamy having a slightly larger effect. Overall, however, the odds of ethnic exogamy hardly change. Furthermore, Latino ethnic exogamy does not change substantially across the two time periods when limiting the analysis to comparable ethnic groups.

For both Asian panethnic categories, controlling for birthplace and language endogamy substantially increases the odds of ethnic exogamy. In the ACS data, the odds of ethnic exogamy are roughly 75\% as high as the odds of ethnic endogamy for both East/Southeast Asians and South Asians once I control for both language and birthplace endogamy. Controlling for language endogamy produces larger changes than controlling for birthplace endogamy, but both variables play a role. The odds of ethnic exogamy for East/Southeast Asians have also increased over time in models that account for birthplace and language endogamy. Using comparable ethnic groups, the odds of ethnic exogamy for East/Southeast Asians were about half those of ethnic endogamy in 1980, but have grown to nearly 75\% by the more recent time period.

For both Latinos and East/Southeast Asians, ethnic exogamy is more likely in models that include all ethnic groups from the ACS data rather than just the groups that were available in the Census 1980 data. These results suggest greater ethnic exogamy among these more recent and smaller groups.

Figure \ref{fig:exog-lolly} shows the important role that birthplace and language endogamy play in panethnic intermarriage. In actuality, we observe similar odds of ethnic exogamy among Asians and Latinos. However, the low odds of ethnic exogamy for Asians are largely a function of the high level of diversity in birthplace and language between Asian ethnic groups. These barriers do not exist for Latinos. Controlling for these factors reveals a stronger affinity for panethnicity among Asians than Latinos in the absence of birthplace and language barriers.

Figure \ref{fig:changes-intermar} shows the odds of East/Southeast Asian and Latino ethnic exogamy in comparison to the odds of ethnoracial exogamy more broadly. All results in Figure \ref{fig:changes-intermar} are from the model that accounts for birthplace and language endogamy. East/Southeast Asian ethnic exogamy stands out as far more likely than any form of interracial marriage. Latino ethnic exogamy, on the other hand, is comparable in likelihood to several forms of interracial marriage. When limiting the analysis to comparable ethnic groups over time, White-Latino intermarriage was less likely than Latino ethnic exogamy in 1980, but has increased in frequency substantially and is now slightly more likely than Latino ethnic exogamy. When the analysis is expanded to all Latino ethnic groups, Latino ethnic exogamy remains sightly more likely than White-Latino exogamy, but the magnitudes are similar.

\begin{figure}
\centering
\includegraphics{main_files/figure-latex/changes-intermar-1.pdf}
\caption{\label{fig:changes-intermar}Odds of ethnoracial exogamy relative to endogamy across two time periods. Results are based on models that control for age differences, educational differences, and birthplace and language endogamy. Values are sorted by ethnoracial exogamy in 1980. Arrows show the change across the two time periods based on comparable sets of ethnic groups. Results for American Indian/Alaska Native intermarriage are excluded due to sampling variability.}
\end{figure}

Figure \ref{fig:changes-intermar} shows a relatively low odds of intermarriage between South Asians and East/Southeast Asians in both time periods. In the more recent ACS data, the odds of an East/Southeast Asian-South Asian intermarriage were only 14\% as high as the odds of ethnic endogamy, making it one of the less likely forms of intermarriage and justifying the decision to separate out these panethnic blocs empirically. The figure also shows significant declines in South Asian interracial marriage with all other groups over the time period. This finding, however, should be viewed with caution. In the earlier time period, the only South Asian group available was Asian Indian and research has shown that respondents sometimes confuse the categories of ``Asian Indian'' and ``American Indian'' in race responses \citep{liebler_american_2004}. Given the novel character of the Asian Indian category in 1980, it is likely that such misreporting has gone down over time. This may contribute to the overall decline in interracial marriage over time, if American Indians are less endogamous, on average, than Asian Indians.\footnote{Although not shown here, I also found that the odds of intermarriage between American Indian/Alaska Natives and South Asians was roughly 1.9 to 1 in 1980 but declined to 0.103 to 1 in the later ACS data. This result suggests that such race reporting errors played a greater role in 1980 than the later data.}

\hypertarget{heterogeneity-in-panethnicity-across-groups}{%
\subsection{Heterogeneity in Panethnicity across Groups}\label{heterogeneity-in-panethnicity-across-groups}}

The preceding models used a single ethnic exogamy term for each panethnic category such that the odds of ethnic exogamy are assumed to be the same regardless of which specific groups are paired. I now turn to models of the ACS data that relax this assumption and fit terms specific to each combination of ethnic groups within the matrix of ethnic groups that make up the three panethnic categories. Additionally, these models also include ethnic group specific parameters for intermarriage with Whites and Blacks. By necessity, these models are limited to the largest ethnic groups within each panethnic category.

Figure \ref{fig:ethnic-heat-map} shows the group-specific odds of ethnic exogamy for the five East/Southeast Asian ethnic groups and the nine Latino ethnic groups for which I could fit models. I also provide a dendrogram that indicates the relative distance between each of these ethnic groups. The dendrogram is calculated by treating the inverse of the odds ratio as a measure of distance and measuring unweighted average distances.

\begin{figure}
\centering
\includegraphics{main_files/figure-latex/ethnic-heat-map-1.pdf}
\caption{\label{fig:ethnic-heat-map}Odds of ethnic exogamy relative to ethnic endogamy between ethnic groups in ACS 2014-2018 data. The upper panel shows East/Southeast Asian ethnic groups and the lower panel shows Latino ethnic groups. Dendrograms on the right are based on hierarchical clustering using unweighted average distances where distance was measured by the inverse of these odds ratios. Results are based on models that control for age differences, education differences, and language and birthplace endogamy.}
\end{figure}

The top panel of Figure \ref{fig:ethnic-heat-map} shows no social distance between the three East Asian groups of Chinese, Korean, and Japanese. The point estimates suggest that, holding language and birthplace endogamy constant, the odds of ethnic exogamy are slightly higher than ethnic endogamy among these groups. However, none of these odds ratios are statistically distinguishable from one using a p-value cutoff of 0.05.

Social distance remains between these East Asian groups and the Vietnamese and Filipino ethnic groups, with the exception of the Vietnamese-Chinese case for which there is no barrier to ethnic exogamy. These results indicate some regional distinction in panethnic intermarriage among East/Southeast Asians. Filipinos, in particular, have the lowest odds of ethnic exogamy with all of the other Asian groups, suggesting a stronger boundary separating Filipinos from wider East/Southeast Asian panethnicity. This may reflect the unique Spanish cultural inheritance of the Philippines. To more formally test this hypothesis, I included a specific dummy variable for Latino-Filipino intermarriage and found that the odds of Latino-Filipino intermarriage were about double the odds of intermarriage between Latinos and other East/Southeast Asian groups.

The bottom panel of Figure \ref{fig:ethnic-heat-map} shows comparable results for the nine Latino ethnic groups. Overall, the odds ratios are much lower than for Asian ethnic groups, reflecting the overall lower ethnic exogamy among Latinos. The only case with no evidence of barriers to ethnic exogamy was Salvadorian-Guatemalan intermarriage.

Affinities between Latino ethnic groups tend to cluster by region, with higher odds of intermarriage within the three groups of South American, Caribbean, and Central American/Mexican nationalities. The one exception to this pattern is for Cubans, who tend to be about equal distance from both the South American and Caribbean groupings.

Figure \ref{fig:racial-exogamy} shows the odds of intermarriage with Whites and Blacks for each of the East/Southeast Asian and Latino ethnic groups. For comparison, the ethnic exogamy odds ratios for a given ethnic group are also shown with small grey dots.

\begin{figure}
\centering
\includegraphics{main_files/figure-latex/racial-exogamy-1.pdf}
\caption{\label{fig:racial-exogamy}Odds of exogamy with Whites and Blacks for major Latino and East/Southeast Asian ethnic groups. For reference, odds of ethnic exogamy to major ethnic groups within the same category are shown by small grey dots. Models are sorted based on the odds ratio of exogamy with Whites. All models control for age differences, education differences, and language and birthplace endogamy.}
\end{figure}

Variation in the likelihood of interracial marriage across ethnic groups is more pronounced for Latino ethnic groups than for East/Southeast Asian groups. Among Latinos, Peruvians and Colombians are the most likely to intermarry with Whites with an odds about 60\% as high as ethnic endogamy. At the other end of the spectrum, the odds ratio of Dominican-White intermarriage is just under 25\%. For the five East/Southeast Asian groups, the odds ratios range from 25\% to 42\%.

For all East/Southeast Asian and Latino ethnic groups, intermarriage with Whites is more likely than intermarriage with Blacks. The odds of intermarriage with Blacks are uniformly low for all five East/Southeast Asian groups. The odds of intermarriage are also low for most Latino groups, with the exception of Dominicans and Puerto Ricans. For Dominicans, the odds of intermarriage with Blacks are almost identical to the odds of intermarriage with Whites. For Puerto Ricans, the odds of intermarriage with Blacks are the highest of any Latino ethnic group and only slightly less likely than the odds of intermarriage with Whites. These differences likely reflect the higher prevalence of Afrodescent in Puerto Rican and Dominican populations and the racialization of members of these populations as Black within the US.

The relative placement of ethnic exogamy odds ratios for each ethnic group in Figure \ref{fig:racial-exogamy} is also telling. For all five Asian ethnic groups, every ethnic exogamy parameter is greater than the odds of intermarriage with Whites. For the Latino groups, these ethnic exogamy parameters are frequently in-between the odds of White and Black intermarriage. For example, both Peruvians and Cubans are more likely to outmarry to a White person than to intermarry with any of the other Latino groups here. Mexicans are only slightly more likely to outmarry to most other Latino ethnic groups as they are to Whites, and are substantially less likely in two cases (Dominicans and Cubans).

Overall, Figure \ref{fig:racial-exogamy} implies very different patterns of intermarriage for Latino and East/Southeast Asian ethnic groups. East/Southeast Asian ethnic groups follow the same general pattern. Panethnic intermarriage is more likely than intermarriage with Whites or Blacks, and the barriers to intermarriage with Blacks are far more substantial than the barriers to intermarriage with Whites. Furthermore the likelihood of intermarriage with Whites and Blacks is relatively similar across ethnic groups. The consistency of this pattern speaks to a broad panethnic pattern of intermarriage for East/Southeast Asians, even given some regional and national variation.

On the other hand, the results for Latino groups are characterized by ethnic-specific heterogeneity. In no case do we see a clear separation in which ethnic exogamy is preferred to outmarriage with Whites. We also see large differences in the tendency to intermarry with Whites and Blacks. These results are not consistent with a broad tendency toward panethnicity among Latinos. Patterns of intermarriage for Latino ethnic groups vary substantially by the ethnic group under consideration.

\hypertarget{conclusions}{%
\section{Conclusions}\label{conclusions}}

This article analyzed the evidence for panethnicity in the partner choices that individuals make in marriage. What evidence do I find for panethnic intermarriage among Asians and Latinos and has this tendency changed over time?

Answering this question is complicated by the diversity of birthplace and language among and within Asian and Latino ethnic groups due to continuing immigration from abroad. Theoretically, such diversity could affect panethnic intermarriage in different ways depending on whether this diversity is greater between rather than within ethnic groups belonging to the same panethnic category. In actuality, it serves as a barrier to panethnic intermarriage among Asians, but not Latinos.

Accounting for birthplace and language endogamy, I see strong affinity between East/Southeast Asian ethnic groups and between South Asian ethnic groups, respectively. In both cases, the odds of ethnic exogamy are about 75\% as high as the odds of ethnic endogamy in the more recent 2014-2018 ACS data. In the case of East/Southeast Asians, I can compare this value to the same value in Census 1980 where the odds were only about 50\% as high, thus indicating growing panethnic affinity over time. The odds of ethnic exogamy among Asians are also much higher than the odds of any form of interracial marriage.

Consistent with prior work, the results demonstrate little affinity between East/Southeast Asians and South Asians. Scholars have raised the question of whether panethnic Asian coalitions in the US truly include South Asians \citep{kibria_not_1996}. In terms of the interpersonal panethnicity measured here, the answer is a resounding no. What emerges instead are two distinct ``melting pots'' of panethnic affinity among Asian populations.

For Latinos, I observe much lower odds of ethnic exogamy. Even after controlling for birthplace and language endogamy, the odds of ethnic exogamy are only about 25\% as high as the odds of ethnic endogamy in both time periods. In comparison, the odds of Latino-White intermarriage are very similar to the odds of ethnic exogamy, suggesting that relative to other forms of exogamy, panethnic intermarriage is not a strong force.

Prior work has often focused only on a few Asian and Latino ethnic groups. In the ACS data, I am able to include a wide variety of ethnic groups and find that when I use more groups, the odds of ethnic exogamy are slightly higher for both East/Southeast Asians and Latinos. I also examine more specific affinities among ethnic groups within the same panethnic category. The results support the more general conclusions, but show some tendency towards regional affinities in both cases. The results also show greater ethnically-based heterogeneity in the intermarriage patterns of Latinos.

Why is panethnic affinity stronger among Asian ethnic groups than among Latino ethnic groups? Consistent with prior research, the results here suggest the relative importance of the structural rather than cultural factors that encourage panethnicity \citep{lopez_panethnicity_1990}. Specifically, prior work has noted the greater tendency for Asians (and in particular East/Southeast Asians) to be racialized as ``Asian'' due to more phenotype similarity and a broad panethnic ``model minority'' stereotype \citep{lopez_panethnicity_1990, kibria_construction_1997, rosenfeld_salience_2001}. My results are consistent with that argument and suggest that racialization also affects this most interpersonal form of panethnicity. This is not to say that racialization plays no role for Latinos, but rather due to greater phenotype diversity as well as other structural differences, the racialization of Latinos has been more contested \citep{rodriguez_changing_2000a, frank_latino_2010a}.

Scholars often treat intermarriage not only as a direct measure of social distance between groups but also as a mechanism for the further breakdown of barriers between groups, because intermarriage will generated progeny of mixed identification in the next generation \citep{gordon_assimilation_1964}. However, group size complicates any attempt to analyze these features of intermarriage simultaneously. I have used models here that remove the issue of group size from consideration when estimating the odds of ethnoracial intermarriage. Such models do a much better job at estimating the underlying affinity between groups. However, the actual frequency of interethnic and interracial marriages will depend much more heavily on group size. For example, because Asian ethnic groups tend to be small within the larger US population, the actual frequency of Asian-White intermarriages will likely be more common than interethnic marriages among Asians, even if Asian individuals prefer the latter type of union. This issue complicates our understanding of how the growth of interethnically married couples and their progeny will affect future racial boundaries.

Understanding the future is further complicated by continuing migration to the US. The strong panethnic affinity I observe among East/Southeast Asian and South Asian ethnic groups only emerges net of the strong tendency toward birthplace and language endogamy. In a situation in which all members of these groups are English-speaking native-born individuals, we would expect such affinities to emerge. In actuality, such affinities are suppressed by the high birthplace and language diversity resulting from continued migration. The future strength of panethnic intermarriage depends very much on future patterns of immigration.



\renewcommand\refname{References}
\bibliography{project.bib}
\end{document}

\processdelayedfloats
